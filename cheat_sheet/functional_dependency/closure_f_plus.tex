\vspace{-0.3cm}
\subsection*{Functional Dependency and Closure}
\vspace{-0.1cm}

\noindent
Functional Dependency (FD) is like a rule that says one piece of data (like a person's name) can tell you another piece of data (like their birthdate). If you know the rule applies consistently, then you know the birthdate for each name without fail.

\vspace{-0.3cm}
\subsection*{Closure Set \( F+ \)}
\vspace{-0.1cm}

\noindent
The closure \( F+ \) is like knowing all the possible things you can find out if you know a starting piece of information under the rules of FD. For example, if knowing someone's name tells you their birthdate, and knowing their birthdate tells you their astrological sign, then knowing a name ultimately tells you both birthdate and astrological sign—that's the closure.

\noindent
\textbf{Example:} If you have a functional dependency:
\[ \text{Name} \rightarrow \text{Birthdate} \]
and another:
\[ \text{Birthdate} \rightarrow \text{Astrological Sign} \]
Then the closure of Name would be:
\[ \text{Name}+ = \{ \text{Name, Birthdate, Astrological Sign} \} \]
