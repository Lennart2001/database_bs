\vspace{-0.3cm}
\subsection*{Block Access Policies}
\vspace{-0.1cm}

\noindent
Block access policies are strategies used primarily in disk scheduling. They determine the order in which disk I/O requests are processed, which can significantly impact the overall performance of the system.

\noindent
\textbf{FIFO (First-In-First-Out)}: The oldest request in the queue is processed first.
Simple to implement. Used in scenarios where fairness is important, and the load is evenly distributed.
Can lead to suboptimal performance if older requests are not as critical as newer ones.

\noindent
\textbf{SJF (Shortest Job First)}: Requests are processed based on the length of the job, with shorter jobs being given priority.
Minimizes the average waiting time and is efficient for systems with varying job sizes.
Can lead to starvation of longer jobs and is difficult to implement, as predicting job length is challenging.

\noindent
\textbf{LOOK}: The disk arm moves in one direction, fulfilling all requests until there are no more in that direction, then reverses its direction.
Reduces the movement of the disk arm, lowering the seek time compared to FIFO.
If requests are heavily skewed to one end, it can lead to longer waiting times for some requests.
