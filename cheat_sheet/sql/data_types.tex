\vspace{-0.3cm}
\subsection*{Data Types}
\vspace{-0.1cm}

\begin{itemize}[noitemsep,leftmargin=*]
\leftskip-\dimexpr\leftmargin %%%
    \item[]{\textbf{INT}: A normal-size integer that can be signed or unsigned.}
    \item[]{\textbf{VARCHAR(n)}: A variable-length string with a maximum length of `n` characters.}
    \item[]{\textbf{TEXT}: A text column with a maximum length of 65,535 characters.}
    \item[]{\textbf{DATE}: A date, formatted as YYYY-MM-DD.}
    \item[]{\textbf{TIMESTAMP}: A timestamp, combining a date and a time.}
    \item[]{\textbf{FLOAT(p, d)}: A floating-point number with a precision `p` and a scale `d`.}
    \item[]{\textbf{DOUBLE(p, d)}: A double precision floating-point number.}
    \item[]{\textbf{DECIMAL(p, d)}: An exact fixed-point number.}
    \item[]{\textbf{BOOLEAN}: A true or false value.}
    \item[]{\textbf{BLOB}: A binary large object that can hold a variable amount of data.}
    \item[]{\textbf{CHAR(n)}: A fixed-length non-binary string.}
    \item[]{\textbf{BIGINT}: A large integer that can be signed or unsigned.}
    \item[]{\textbf{ENUM(val1, val2, ...)}: A string object that can only have one value, chosen from a list of possible values.}
    \item[]{\textbf{SET(val1, val2, ...)}: A string object that can have zero or more values, chosen from a list of possible values.}
\end{itemize}